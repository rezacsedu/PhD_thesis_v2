\begin{appendices}
\section*{Appendices}
The doctoral candidate participated in several projects, teaching, grant proposal writing, thesis supervision, publishing scientific publications, review works, and contributing to open-sources. 

\iffalse
\subsection*{Project works}
The doctoral candidate participated in the following research projects at Fraunhofer FIT. Besides he was involved in other project works jointly done at Fraunhofer FIT and RWTH Aachen University:

\begin{enumerate}[noitemsep]
    \item German Medical Informatics BMBF projects
    \item DEMETER: Building an Interoperable, Data-Driven, Innovative and Sustainable European Agri-Food Sector~(Grant agreement ID: 857202)
    \item Data Scientist Basic Level~(900362). 
\end{enumerate}  

\subsection*{Project proposal writing}
The doctoral candidate participated in the following research projects at Fraunhofer FIT during his studies. Besides he was involve in collaborative projects between Fraunhofer FIT and RWTH Aachen University: 

\begin{enumerate}[noitemsep]
    \item H2020- Digital diagnostics – developing tools for supporting clinical decisions by integrating various diagnostic data~(SC1-BHC-06-2020)
    \item CORD
    \item POLAR.
\end{enumerate}  
\fi 

\subsection*{Scientific publications~(published + under review)}
% Big Data
I have published several research articles in top-ranked conferences and journals in which I proposed and implemented various data-driven and engineering solutions to core and applied research problems, covering the area of machine learning, computer vision, data engineering, bioinformatics, and natural language processing~(NLP). Subsequently, my research articles have reached 451 \textbf{Google Scholar} citations\footnote{\url{https://scholar.google.de/citations?user=LTTNF64AAAAJ&hl=en}}. 

\hspace*{5mm} Various types of data from numerous sources such as ubiquitous devices, social networks, sensing devices, and streaming services are being produced at unprecedented speed, making it difficult to capture, manage and process them with low latency. These data often called `big data' has characteristics like high volume, high velocity or high variety. Therefore, in order to analyse those data towards revealing patterns, trends, and associations, we need computationally faster and efficient solutions. Based on this motivations, I employed efficient machine learning and data mining methods to such large-scale datasets to find valuable insights, which reflected in the following journal papers:

\begin{enumerate}[noitemsep]
	\item {\bf Md. Rezaul Karim}, Michael Cochez, Oya Beyan, C. Farhan Ahmed, and Stefan Decker, ``Mining Maximal Frequent Patterns in Big Transactional Databases and Dynamic Data Streams: A Spark Based Approach", \emph{Information Sciences}~(Impact Factor: 5.524), Vol-432, pp 278-300, March 2018.
	
	\faLink~\url{https://www.sciencedirect.com/science/article/pii/S002002551731126X}
	
	\faGithub~\url{https://github.com/rezacsedu/Mining-Maximal-Frequent-Pattern-Spark}
	
	\item Ashfaq Khan, \textbf{Md. Rezaul Karim}, Yangwoo Kim, ``A Two-Stage Big Data Analytics Framework with Real World Applications Using Spark Machine Learning and Long Short-Term Memory Network", \emph{Symmetry}~(Impact Factor: 2.51), 10(10):~485, October 2018. 
	
	\faLink~\url{https://www.mdpi.com/2073-8994/10/10/485}
	
	\faGithub~\url{https://github.com/rezacsedu/2Stage-Big-Data-Analytics-SparkML-LSTM}
	
	\item Ashfaq Khan, \textbf{Md. Rezaul Karim}, Yangwoo Kim, ``A Scalable and Hybrid Intrusion Detection System Based on  Convolutional-LSTM Network", \emph{Symmetry}~(Impact Factor: 2.51), 2019, 11, 583. 
	
	\faLink~\url{https://www.mdpi.com/2073-8994/11/4/583}
	
	\faGithub~\url{https://github.com/rezacsedu/Intrusion-Detection-Spark-Conv-LSTM}
\end{enumerate}

% Cancer Genomics
Cancer is one of the deadliest diseases caused by abnormal behaviours of genes that control cell division and growth. Since an appropriate diagnosis depends on accurate prediction of cancer type, it is crucial to analyze genomics data and clinical outcomes before recommending necessary treatments. Analyzing such data can provide profound insights to reveal genetic predispositions of cancer before it grows. Based on this motivation, I proposed efficient analytical methods based on supervised machine learning techniques on genomics data for the diagnoses of cancer, which reflected in the published papers in top journals: 

\begin{enumerate}[noitemsep]
	\item \textbf{Md. Rezaul Karim}, Ashiqur Rahman, Stefan Decker, and Oya Beyan, ``A Neural Ensemble Method for Cancer Type Prediction Based on Copy Number Variations", \emph{Neural Computing and Applications}~(Impact Factor: 4.680), November 30, 2019. 
	
	\faLink~\url{https://link.springer.com/article/10.1007/s00521-019-04616-9}
	
	\faGithub~\url{https://github.com/rezacsedu/Neural-ensemble-method-for-cancer-prediction} 
	
	\item \textbf{Md. Rezaul Karim}, Stefan Decker, Oya Beyan, ``Multimodal Autoencoders for Prognostically Relevant Subtypes and Survival Prediction for Breast Cancer", \emph{IEEE Access}~(Impact Factor: 4.098), DOI: 10.1109/ACCESS.2019.2941796, September 2019.
	
	\faLink~\url{https://ieeexplore.ieee.org/document/8839793}
	
	\faGithub~\url{https://github.com/rezacsedu/Multimodal-autoencoder-for-breast-cancer} 
	
	\item {Alokkumar Jha, {\bf Md. Rezaul Karim}, Dietrich Rebholz-Schuhmann, and Ratnesh Sahay, ``Discovering Biomarker and Pathway for Gynecological Cancers", \emph{Journal of Biomedical Semantics}~(Impact Factor: 2.28), 8(1), September 2017, DOI: 10.1186/s13326-017-0146-9.} 
	
	\faLink~\url{https://jbiomedsem.biomedcentral.com/articles/10.1186/s13326-017-0146-9}
\end{enumerate}

% Bioinformatics
The exponential growth of the amount of biomedical and biological data available today requires both efficient information storage/management and computational methods for the extraction of useful information towards better and improved healthcare. Efficient tools and methods are capable of transforming all these heterogeneous data into biological knowledge about the underlying mechanism. Based on these motivations, I applied unsupervised machine learning techniques and show how to extract knowledge to take informed decision, which reflected in the published papers in journals:

\begin{enumerate}[noitemsep]
	\item \textbf{Md. Rezaul Karim}, Oya Beyan, Achille Zappa, Ivan G. Costa, Dietrich Rebholz-Schuhmann, Michael Cochez, and Stefan Decker, ``Deep Learning-based Clustering Approaches for Bioinformatics", \emph{Briefings in Bioinformatics}~(Impact Factor: 9.101), February, 2020.
	
	\faLink~\url{https://academic.oup.com/bib/advance-article/doi/10.1093/bib/bbz170/5721075}

	\faGithub~\url{https://github.com/rezacsedu/Deep-learning-for-clustering-in-bioinformatics}
		
	\item \textbf{Md. Rezaul Karim}, Michael Cochez, Oya Beyan, Dietrich Rebholz-Schuhmann, and Stefan Decker, ``Convolutional Embedded Networks for Population Scale Clustering and Bio-ancestry Inferencing", \emph{IEEE/ACM Transactions on Computational Biology and Bioinformatics}~(Impact Factor: 2.428), Accepted and to appear in June 2020.
    
    \faLink~\url{https://ieeexplore.ieee.org/document/8839793}

	\faGithub~\url{https://github.com/rezacsedu/Convolutional-embedded-networks}
\end{enumerate}

% XAI
Deep learning based on neural networks~(DNN) has shown tremendous success in automated decision-making, covering in numerous domains. However, due to the nested non-linear and complex structure, DNN architectures are mostly opaque and perceived as `black box' methods. They not only suffer from a lack of transparency but also cannot reason their underlying decisions. Such opaqueness raises numerous legal, ethical, and practical concerns. AI-based systems have already been utilized in numerous domain such as automated diagnoses, treatment, and prognosis in a clinical setting. 

\hspace*{5mm} However, if we cannot see how a clinical decision is made, we cannot know what impact it will create on a patient, because the day when such AI-guided systems will make life decisions for humans is not very far ahead. Based on this motivations, I developed several predictive models by embeddings different interpretable~(both ante-hoc and post-hoc) logic towards making them explainable, interpretable, and fair. I hope my approaches will be a fruitful contribution, particularly, towards the development of AI-assisted applications and an acceleration of their adoption in the clinical practice: 

\begin{enumerate}[noitemsep]
	\item \textbf{Md. Rezaul Karim}, Jiao Jiao, Michael Cochez, Oya Beyan, and Stefan Decker, ``DeepKneeExplainer: Explainable Knee Osteoarthritis Diagnosis Based on Radiographs and Magnetic Resonance Images", under review in \emph{Artificial Intelligence~(Elsevier)}~(Impact Factor: 4.483), September 2019.

	\faGithub~\url{https://github.com/rezacsedu/DeepKneeOA}
	\item \textbf{Md. Rezaul Karim}, Till Döhmen, S. Decker, O. Beyan, ``DeepCOVIDExplainer: Explainable COVID-19 Predictions Based on Chest X-ray Images", Under review in \emph{IEEE International Conference on Data Science and Advanced Analytics~(DSAA’2020)}. 
	
	\faGithub~\url{https://github.com/rezacsedu/DeepCOVIDExplainer}
	
	\item \textbf{Md. Rezaul Karim}, Michael Cochez, Oya Beyan, Stefan Decker, and Christoph Lange-Bever, ``OncoNetExplainer: Explainable Predictions of Cancer Types Based on Gene Expression Data", Proc. of \emph{IEEE International Conference on Bioinformatics and Bioengineering~(BIBE 2019)}, October 28-30, 2019, Athens, Greece.
	
	\faLink~\url{https://ieeexplore.ieee.org/document/8941872}

	\faGithub~\url{https://github.com/rezacsedu/Explainable-cancer-type-prediction}
\end{enumerate}

% Semantics 
Semantic Web~(SW) technologies are increasingly applied to several domains. For example, A knowledge graph~(KG) can be built to integrate domain-specific information, facts, ans evidences into an ontology and applies a reasoner to derive new knowledge. On the other hand, life sciences is an early adopter of SW technologies. Based on these motivations, I prepared computational resources by employing SW technologies, which includes preparing knowledge graphs, preparing RDF triple store, formulating SPARQL queries to query the KG, and analyse the data. Besides, I published several papers towards making the research data FAIR~(findable, accessible, interoperable, and reproducible): 

\begin{enumerate}[noitemsep]
	\item \textbf{Md. Rezaul Karim}, Michael Cochez, Joao Jares, Oya Beyan, Stefan Decker, ``Drug-Drug Interaction Prediction Based on Knowledge Graph Embeddings and Convolutional-LSTM Network", Proc. of \emph{ACM International Conference on Bioinformatics, Computational Biology, and Health-informatics~(ACM-BCB)}, Niagara Falls, New York, USA, September 7-10, 2019.
	
	\faLink~\url{https://dl.acm.org/doi/10.1145/3307339.3342161}
	
	\item Oya Beyan, Ananya Choudhury, \textbf{Md. Rezaul Karim}, Michel Dumontier, Stefan Decker, Andre Dekker, ``Distributed Analytics on Sensitive Medical Data: The Personal Health Train", \emph{Data Intelligence}, 2 (2020), 96–107. 
    
    \faLink~\url{http://www.data-intelligence-journal.org/p/39/}

	\faGithub~\url{https://github.com/rezacsedu/Drug-drug-interaction-prediction} 
	\item Lars C. Gleim, \textbf{Md. Rezaul Karim}, Lukas Zimmermann, Oliver Kohlbacher, Holger Stenzhorn, Stefan Decker and Oya Beyan, ``Enabling Ad-hoc Reuse of Private Data Repositories through Schema Extraction", \emph{Journal of Biomedical Semantics}~(Impact Factor: 2.28), 2020. 
	
	\faGithub~\url{https://github.com/PersonalHealthTrainGermany/schemaExtraction}
	
    \item {{\bf Md. Rezaul Karim}, Ratnesh Sahay, and Dietrich Rebholz-Schuhmann, ``Improving Data Workflow Systems with Cloud Services and Use of Open Data for Bioinformatics Research", \emph{Briefings in Bioinformatics}~(Impact Factor: 9.101), 2017, 1–16, DOI: 10.1093/bib/bbx039.}
    
    \faLink~\url{https://www.ncbi.nlm.nih.gov/pmc/articles/PMC6169675/pdf/bbx039.pdf}
    
    \item \textbf{Md. Rezaul Karim}, Decker S, Oya Beyan, ``A Distributed Analytics Platform to Execute FHIR-based Phenotyping Algorithms", \emph{$11^{th}$ International Conference on Semantic Web Applications \& Tools for Healthcare and Life Sciences~(SWAT4HCLS'2018)}, Antwerp, Belgium, 3-6 December 2018.

    \item \textbf{Md. Rezaul Karim}, Heinrichs M, and Oya Beyan, ``Towards a FAIR Sharing of Scientific Experiments: Improving Discoverability and Reusability of Dielectric Measurements of Biological Tissues'', \emph{$10^{th}$ International Conference on Semantic Web Applications and Tools for Healthcare \& Life Sciences~(SWAT4HCLS,2017), Rome, Italy, 4-7 December 2017}.
\end{enumerate}

% NLP
Exponential growths of social media and micro-blogging sites not only provide platforms for empowering freedom of expressions and individual voices, but also enables people to express anti-social behaviour like online harassment, cyberbullying, and hate speech. Numerous works have been proposed to utilize these data for social and anti-social behaviours analysis, document characterization, and sentiment analysis by predicting the contexts mostly for highly resourced languages like English. 

\hspace*{5mm} However, some languages are under-resources, e.g., South Asian languages like Bengali, Tamil, etc. that lack of computational resources for natural language processing~(NLP). Based on this motivation, I provide classification benchmarks for under-resourced \textit{Bengali} language based on 3 datasets of hates, commonly used topics, and opinions for hate speech detection, document classification, and sentiment analysis. We built the largest Bengali word embedding models to date based on 250 million articles:

\begin{enumerate}[noitemsep]
	\item \textbf{Md. Rezaul Karim}, Bharathi Raja Chakravarti, Mihael Arcan, John P. McCrae, and Michael Cochez, ``Classification Benchmarks for Under-resourced Bengali Language based on Multichannel Convolutional-LSTM Network", Proc. of the \emph{$7^th$ IEEE International Conference on Data Science and Advanced Analytics~(DSAA,2020)}, 6-9 October 2020, Sydney, Australia.
	\faGithub~\url{https://github.com/rezacsedu/BengFastText}
\end{enumerate}

\subsection*{Review works and scientific community involvement}%{\faMailReplyAll}
My research contributions were also recognised by the scientific communities, where I made significant voluntary contributions, by reviewing research articles for journals and conferences and acting as program committee members for conferences and workshops. I was invited to be a program committee for the following events:

\begin{itemize}[noitemsep]
    \item The European Conference on Machine Learning and Principles \& Practice of Knowledge Discovery in Databases~(ECML-PKDD)
    \item International World Wide Web Conference~(WWW)
    \item Extended Semantic Web Conference~(ESWC) 
    \item International Semantic Web Conference~(ISWC) 
    \item International Conference on Web and Social Media~(ICWSM) 
    \item The ACM Web Science Conference~ACM WebSci )
    \item International Conference on Knowledge Capture~ACM K-CAP).
\end{itemize}

The editors of the following prestigious and top-ranked journals invited me to review scientific articles: 

\begin{itemize}[noitemsep]
    \item IEEE/ACM Transactions on Computational Biology and Bioinformatics
    \item WIREs Journal of Data Mining and Knowledge Discovery
    \item Journal of Expert Systems with Applications 
    \item Journal of Frontiers of Computer Science
    \item Journal of Briefings in Bioinformatics
    \item Journal of Biomedical Semantics
    \item Journal of Cloud Computing
    \item Journal of Medical Systems
    \item Packt Publishing Ltd.~(UK)
    \item Journal of IEEE Access 
    \item Semantic Web Journal.
\end{itemize}

The editors of the following academic and professional book publishers invited me to review the books and the book proposals: 

\begin{itemize}[noitemsep]
    \item Packt Publishing Ltd.~(UK)
    \item Elsevier academic publisher. 
\end{itemize}

\subsection*{Contributions to open-source and reproducible science %(technical advisor)
}%{\faGithub}
I made source-codes of my several publications open access for the research communities. Besides, I contributed to several open-source implementations, which include feature implementations, filing issues, pull request, answering to technical questions: 

\begin{itemize}[noitemsep]
    \item GitHub
    \item GitLab
    \item Apache Spark 
    \item LIME
    \item ExplainX
    \item SHAP
    \item StackOverflow.
\end{itemize}

There are 302 repositories in my GitHub profile\footnote{\url{https://github.com/rezacsedu}} with more than 500 starts. Besides, I made several research datasets openly accessible for the researchers that came with GitHub, Figshare\footnote{\url{https://figshare.com/}}, or sharing via Fraunhofer owncloud.

\end{appendices}